\documentclass[author-year, review]{elsarticle} %review=doublespace preprint=single 5p=2 column
%%% Begin My package additions %%%%%%%%%%%%%%%%%%%
\usepackage[hyphens]{url}
\usepackage{lineno} % add 
%\linenumbers % turns line numbering on 
\bibliographystyle{elsarticle-harv}
\biboptions{sort&compress} % For natbib
\usepackage{graphicx}
\usepackage{booktabs} % book-quality tables
%% Redefines the elsarticle footer
\makeatletter
\def\ps@pprintTitle{%
 \let\@oddhead\@empty
 \let\@evenhead\@empty
 \def\@oddfoot{\it \hfill\today}%
 \let\@evenfoot\@oddfoot}
\makeatother

% A modified page layout
\textwidth 6.75in
\oddsidemargin -0.15in
\evensidemargin -0.15in
\textheight 9in
\topmargin -0.5in
%%%%%%%%%%%%%%%% end my additions to header



\usepackage[T1]{fontenc}
\usepackage{lmodern}
\usepackage{amssymb,amsmath}
\usepackage{ifxetex,ifluatex}
\usepackage{fixltx2e} % provides \textsubscript
% use upquote if available, for straight quotes in verbatim environments
\IfFileExists{upquote.sty}{\usepackage{upquote}}{}
\ifnum 0\ifxetex 1\fi\ifluatex 1\fi=0 % if pdftex
  \usepackage[utf8]{inputenc}
\else % if luatex or xelatex
  \usepackage{fontspec}
  \ifxetex
    \usepackage{xltxtra,xunicode}
  \fi
  \defaultfontfeatures{Mapping=tex-text,Scale=MatchLowercase}
  \newcommand{\euro}{€}
\fi
% use microtype if available
\IfFileExists{microtype.sty}{\usepackage{microtype}}{}
\usepackage{graphicx}
% We will generate all images so they have a width \maxwidth. This means
% that they will get their normal width if they fit onto the page, but
% are scaled down if they would overflow the margins.
\makeatletter
\def\maxwidth{\ifdim\Gin@nat@width>\linewidth\linewidth
\else\Gin@nat@width\fi}
\makeatother
\let\Oldincludegraphics\includegraphics
\renewcommand{\includegraphics}[1]{\Oldincludegraphics[width=\maxwidth]{#1}}
\ifxetex
  \usepackage[setpagesize=false, % page size defined by xetex
              unicode=false, % unicode breaks when used with xetex
              xetex]{hyperref}
\else
  \usepackage[unicode=true]{hyperref}
\fi
\hypersetup{breaklinks=true,
            bookmarks=true,
            pdfauthor={},
            pdftitle={Beyond Mechanistic Models: The case for machine learning in ecological management},
            colorlinks=true,
            urlcolor=blue,
            linkcolor=magenta,
            pdfborder={0 0 0}}
\urlstyle{same}  % don't use monospace font for urls
\setlength{\parindent}{0pt}
\setlength{\parskip}{6pt plus 2pt minus 1pt}
\setlength{\emergencystretch}{3em}  % prevent overfull lines
\setcounter{secnumdepth}{0}
% Pandoc toggle for numbering sections (defaults to be off)
\setcounter{secnumdepth}{0}
% Pandoc header



\begin{document}
\begin{frontmatter}
  \title{Beyond Mechanistic Models: The case for machine learning in ecological
         management}
  \author[cstar]{Carl Boettiger\corref{cor1}}
  \author[cstar]{Marc Mangel}
  \author[noaa]{Stephan B. Munch}
  \ead{cboettig@ucsc.edu}
  \cortext[cor1]{Corresponding author, cboettig@ucsc.edu}
  \address[cstar]{Center for Stock Assessment Research, Department of Applied Math and Statistics, University of California, Mail Stop SOE-2, Santa Cruz, CA 95064, USA}
  \address[noaa]{Southwest Fisheries Science Center, National Oceanic and Atmospheric Administration, 110 Shaffer Road, Santa Cruz, CA 95060, USA}
 \end{frontmatter}


\section{Abstract}

\section{Introduction}

Process-based models built on modern machine learning approaches offer a
powerful and underappreciated approach for ecological forecasting and
management. We present a novel example of how one such example -
non-parametric Gaussian processes, can be adapted to the context of
management and decision-making.

While effective machine learning algorithms have speckled the
literature, most theoretical ecologists in particular have been wary of
these `black-box' approaches in favor of simple, mechanistic models.
While a blind application of machine learning approaches would indeed be
dangerous grounds to stake either ecological understanding or management
practice, such concerns are not grounds to reject such methods out of
hand. Rather, further attention is warranted to identify the
circumstances in which machine-learning approaches will most likely
outperform existing more mechanistic approaches, or help crack difficult
management questions where ecological or dynamical complexity has caused
traditional approaches to founder.

Here we illustrate how a machine learning model can be implemented in
place of simple mechanistic model in a decision theory (optimal control)
problem, and highlight the features of this system that make the
nonparametric approach particularly effective.

Mechanistic models have long been the gold standard of theoretical
modeling in ecology (e.g.~see @Geritz2012 or @Cuddington2013). Only by
understanding the processes involved can we make reliable long term
predictions and build an knowledge of cause and effect that guides the
hypotheses we make, the data we collect, and the management decisions we
make. Process-based models, whether expressed in the language of
mathematics or English, identify the connection between mosquitoes and
the spread of malaria, or greenhouses gases and climate change, guiding
our approach to understand and manage these threats. Despite this
central importance, we argue here that ecologists would do well to give
greater attention to the role non-mechanistic models can play in
ecological management and decision making. The value of these approaches
is greatest in a context where decisions are made over short time
horizons, and updated as new data becomes available.

\subsection{The alternative to process-based models: pattern-based
modeling}

Before we proceed further it would be useful to define our terms and lay
some greater context for the discussion. Throughout, we will distinguish
``mechanistic'' or ``process-based'' models from models that are merely
``correlative'' or ``pattern-based''. Historically pattern-based
modeling meant regression (usually linear regression) and resided in
methodology rooted in statistics departments and was the primary focus
of empirical ecologists, while mechanistic modeling meant dynamical
systems (ODEs, PDEs, SDEs and their discrete kin), residing in
methodology from the mathematics department and was the primary focus of
theoretical ecologists.

We believe these historical divides continue to color much of the
literature today, with theorists more skeptical than empiricists of
statistical methodology and vice versa. Meanwhile, the ground underneath
has shifted. Originally pattern-matching approaches could be critiqued
on the grounds of their simplicity (not everything is linear) while
mechanistic approaches could be critiqued on their tenuous connections
to data -- model parameter values such as death rates, birth rates, etc.
would be estimated in advance and then stuck into the model, rather than
estimated in the context of the model itself. Today both of these
critiques are outdated. Computational power and hierarchical statistical
methods (particularly approximate likelihood or simulation techniques
such as particle filters and approximate Bayesian computing) have
brought the inference richer dynamical systems into their fold, while
pattern-based modeling has spawned an entirely new approach under the
banner of machine learning that has divided the statistics community
(See @Bremman2001, ``The Two Cultures''). Machine learning models can
represent almost arbitrarily complex patterns and incorporate learning
and decision making strategies that have earned them the name
`artificial intelligence.' Yet at the same time they can be far less
transparent their linear regression predecessors, making both astounding
successes and startling biases and failures like their biological
namesake. More ecologists, particularly theorists and modelers, would do
well to learn both how to take advantage of these strengths and
recognize their weaknesses.

Mechanistic modeling emphasizes the importance of capturing the correct
gross properties of a system over tracking minute fluctuations. For
instance, in selecting and parameterizing model of a population of
conservation concern, we may be most interested in getting the long-term
behavior correct -- such as identifying if the dynamics support
persistence of the population -- rather than worrying how well they
reflect the year-to-year fluctuations. We would certainly have good
reason to prefer such a model over alternatives which are irreconcilable
to the most basic biological processes, such as unbounded growth, or
growth curves that do not pass through the origin in a closed system. So
it may come as a surprise to realize that such obviously wrong models
can perform as well or even better than reasonable mechanistic models in
guiding ecological management and decision making.

A simple example comparing the performance of the two modeling
approaches in a quantitative decision problem will help to introduce
these issues.

\section{Models \& Methods}

\begin{figure}[htbp]
\centering
\includegraphics{figure/gp_plot.pdf}
\caption{Graph of the inferred Gaussian process compared to the true
process and maximum-likelihood estimated process. Graph shows the
expected value for the function $f$ under each model. Two standard
deviations from the estimated Gaussian process covariance with (light
grey) and without (darker grey) measurement error are also shown. The
training data is also shown as black points. (The GP is conditioned on
0,0, shown as a pseudo-data point).}
\end{figure}

\begin{figure}[htbp]
\centering
\includegraphics{figure/policies_plot.pdf}
\caption{The steady-state optimal policy (infinite boundary) calculated
under each model. Policies are shown in terms of target escapement,
$S_t$, as under models such as this a constant escapement policy is
expected to be optimal (Reed 1979).}
\end{figure}

\begin{figure}[htbp]
\centering
\includegraphics{figure/sim_plot.pdf}
\caption{Gaussian process inference outperforms parametric estimates.
Shown are 100 replicate simulations of the stock dynamics (eq 1) under
the policies derived from each of the estimated models, as well as the
policy based on the exact underlying model.}
\end{figure}

An example of such a comparison is shown in Figure 1, in which the
optimal management decision is determined by stochastic dynamic
programming algorithm using the biologically (a) plausible and (b) the
implausible model of the population growth.

\section{Discussion}

Three elements contribute to the biologically implausible
phenomenological model performing better in this context:

\subsubsection{1. Relevant state space}

The dynamics occur over a range of state-space in which the biologically
implausible model performs as well or better than the more plausible
model.

This is the most obvious and immediate reason why the shortcomings that
appear to make the model implausible do not make it useless. This effect
is enhanced in our example because the management actions help drive the
system towards rather than away from this region of state-space.

\subsubsection{2. Predictive accuracy}

This application relies only on the predictive accuracy of the model,
not an interpretation of the parameters. Predictive accuracy is not the
goal of all modeling, as ecologists have been observing for as long as
they made models (perhaps none more memorably than @Levins1969).

Mechanistic modeling is at its most powerful not when it is used to
directly forecast future states but when it provides an understanding of
how to approach a problem. SIR-type models from the epidemiological
literature are a good example. While the simplest SIR models have little
predictive power over the outbreak intensity or timing at a particular
location, they provide a powerful understanding of the spread of an
infection in terms of a single, biologically meaningful parameter:
$R_0$, the basic reproductive number. From the model, it becomes clear
that management need not vaccinate every member of the population to
stop the spread, but rather it suffices to vaccinate a sufficient
fraction of the population to reduce $R_0$ below 1.

\subsubsection{3. Time scale for new data}

In the sequential decision making problem we considered, we are
presented with new data after each action. The relevant timescale for
the prediction is thus not the long-term dynamics, which would be wildly
divergent, but the dynamics over this much shorter interval. While
ecologists may be hesitant to base continual management on a model with
obviously inaccurate long-term behavior, engineers tend to consider the
problem in frequency space and gravitate to the opposite position -- a
good control model should prioritize high-frequency accuracy over low
frequency accuracy. The differences in intuition may arise from the
timescales at which each profession can typically adjust their control
variables -- much faster for a control system of a chemical plant than
state policy for a natural resource. Still, the lesson is clear: when
facing repeated management decisions over a short timescale, such as
setting annual harvests of a fishery, it may be more valuable to use a
machine learning algorithm that makes accurate year-ahead predictions
that capture some of the high-frequency fluctuations that appear only as
noise in a mechanistic model of the long-term population dynamics.

\subsection{Advantages of non-mechanistic models}

\begin{enumerate}
\def\labelenumi{\arabic{enumi}.}
\itemsep1pt\parskip0pt\parsep0pt
\item
  Can better use all of the data available.
\end{enumerate}

Incorporating various sources of information into mechanistic models can
be an immensely difficult due to the increased complexity involved. Only
thanks to tandem advances in increasing computational power and
hierarchical statistical methodology have we been able to tackle such
intuitively important complexity (and the potentially new available data
that accompanies it) such as spatial distribution, heterogeneities of
space, time, and individuals, to shift to ecosystem-based approaches
from single-species based approaches. Without the need to formulate
mechanisms, many modern machine learning algorithms can leverage
potential information from all available sources of data directly. The
algorithms can recognize unanticipated or subtle patterns in large data
sets that enable more accurate predictions than mechanistic models that
are formulated at a more macroscopic level. Such approaches depend
critically

\begin{enumerate}
\def\labelenumi{\arabic{enumi}.}
\setcounter{enumi}{1}
\itemsep1pt\parskip0pt\parsep0pt
\item
  Better express ambiguity in regions where data is not available
\end{enumerate}

One of the greatest strengths of mechanistic models is their greatest
weakness as well.

\subsection{Better the devil you know? The dangers of pattern-based
approaches}

When all we have identified is a pattern rather than a mechanism, there
is little way to guarantee that the pattern will not change in the
future. Few examples illustrate this danger more broadly than the
sub-prime mortgage crisis of 2008. Simplifying greatly, the unquenchable
demand for mortgage backed securities had shifted the underlying
dynamics on which the phenomenological models were based. Historical
data indicated that mortgage-backed securities involved little risk. As
demand exhausted the supply of traditional mortgages, it became
profitable to invent new mortgages that could be sold to higher risk
customers, such as those without proof of income. The pattern-based
models did not have enough time or data to learn about the higher
default rate occurring as a result, but remained anchored too in the old
patterns where default rates were low. While a mechanistic formulation
may have anticipated the resulting higher risk, the pattern based model
must learn by seeing it. The economy still struggles to pay off that
lesson.

Of course mechanistic models may also fail to predict sudden structural
changes. It is all too easy to overlook some slowly changing
environmental variable, particularly if temporal changes in a parameter
must be modeled explicitly if they are to be included at all. Machine
learning approaches are frequently less rigid in their assumptions, but
can also less transparent about them. This creates a second weakness in
the machine learning approach.

Mechanistic models are not free from this danger either. In fact they
can be more susceptible to them for the very same reasons. A mechanistic
model has the ability to more convincingly produce out-of-sample
predictions or extrapolation. This may allow us to determine that a
virus will spread or an endangered species will go extinct even before
we have observed

Dangers of biased estimation.

Parametric models are frequently used phenomenologically, rather than
being derived by plausible underlying mechanisms. @Geritz2012 condemn
this practice on the basis of the common misinterpretations that result.
These mistakes arise from ignoring or misinterpreting the biological
meaning of the model parameters. The use of delay equations is a
frequent offender -- for instance, one is hard-pressed to describe a
mechanistic model consistent with Hutchinson's 1948 delay-logistic.
Assuming either scramble competition or interference competition gives
rise to two alternative delay equations, both structurally different
from the original.

\begin{quote}
since the late 1980s there has been some consensus amongst ecologists
that management decisions are best guided by models which are grounded
in ecological theory, and which strike a balance between too much or too
little detail describing the relevant processes (e.g., DeAngelis 1988,
Starfield 1997, Jackson et al. 2000, Carpenter 2003, Nelson et al. 2008)
\end{quote}

@Geritz2012 @Cuddington2013

Managing by rule-of-thumb

Managing by algorithm

\begin{figure}[htbp]
\centering
\includegraphics{figure/posteriors.pdf}
\caption{Histogram of posterior distributions for the estimated Gaussian
Process shown in Figure 1. Prior distributions overlaid.}
\end{figure}

Reed, William J. 1979. ``Optimal escapement levels in stochastic and
deterministic harvesting models.'' \emph{Journal of Environmental
Economics and Management} 6 (4) (dec): 350--363.
doi:10.1016/0095-0696(79)90014-7.

\end{document}


